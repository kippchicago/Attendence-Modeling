\documentclass[sfsidenotes, justified]{tufte-handout}\usepackage[]{graphicx}\usepackage[]{color}
%% maxwidth is the original width if it is less than linewidth
%% otherwise use linewidth (to make sure the graphics do not exceed the margin)
\makeatletter
\def\maxwidth{ %
  \ifdim\Gin@nat@width>\linewidth
    \linewidth
  \else
    \Gin@nat@width
  \fi
}
\makeatother

\definecolor{fgcolor}{rgb}{0.345, 0.345, 0.345}
\newcommand{\hlnum}[1]{\textcolor[rgb]{0.686,0.059,0.569}{#1}}%
\newcommand{\hlstr}[1]{\textcolor[rgb]{0.192,0.494,0.8}{#1}}%
\newcommand{\hlcom}[1]{\textcolor[rgb]{0.678,0.584,0.686}{\textit{#1}}}%
\newcommand{\hlopt}[1]{\textcolor[rgb]{0,0,0}{#1}}%
\newcommand{\hlstd}[1]{\textcolor[rgb]{0.345,0.345,0.345}{#1}}%
\newcommand{\hlkwa}[1]{\textcolor[rgb]{0.161,0.373,0.58}{\textbf{#1}}}%
\newcommand{\hlkwb}[1]{\textcolor[rgb]{0.69,0.353,0.396}{#1}}%
\newcommand{\hlkwc}[1]{\textcolor[rgb]{0.333,0.667,0.333}{#1}}%
\newcommand{\hlkwd}[1]{\textcolor[rgb]{0.737,0.353,0.396}{\textbf{#1}}}%

\usepackage{framed}
\makeatletter
\newenvironment{kframe}{%
 \def\at@end@of@kframe{}%
 \ifinner\ifhmode%
  \def\at@end@of@kframe{\end{minipage}}%
  \begin{minipage}{\columnwidth}%
 \fi\fi%
 \def\FrameCommand##1{\hskip\@totalleftmargin \hskip-\fboxsep
 \colorbox{shadecolor}{##1}\hskip-\fboxsep
     % There is no \\@totalrightmargin, so:
     \hskip-\linewidth \hskip-\@totalleftmargin \hskip\columnwidth}%
 \MakeFramed {\advance\hsize-\width
   \@totalleftmargin\z@ \linewidth\hsize
   \@setminipage}}%
 {\par\unskip\endMakeFramed%
 \at@end@of@kframe}
\makeatother

\definecolor{shadecolor}{rgb}{.97, .97, .97}
\definecolor{messagecolor}{rgb}{0, 0, 0}
\definecolor{warningcolor}{rgb}{1, 0, 1}
\definecolor{errorcolor}{rgb}{1, 0, 0}
\newenvironment{knitrout}{}{} % an empty environment to be redefined in TeX

\usepackage{alltt}
\usepackage{url}
\usepackage[english]{babel}
\usepackage{blindtext}
\usepackage{multirow} 
\usepackage{tikz}
\usetikzlibrary{shapes,arrows}



\title{School Quality Rating Policy (SQRP) Summary}
\author{Pilsen Team}


\makeatother
\IfFileExists{upquote.sty}{\usepackage{upquote}}{}


\begin{document}




\maketitle
\begin{abstract}
The aim of this document and its attendent presentation is to provide clarity around Chicago Public Schools' new performance policy: the School Quality Rating Policy (SQRP).  The SQRP replaces the Performance, Remediation, and Probabation Policy (PRPP).  The two performance polices differ greatly, but the most significant change is that the SQRP relies almost exclusively on NWEA's MAP assessment to measure academic progress rather than the ISAT. MAP-based metrics account for 60-65\% of a school's earned points in the rating system.  The remaining 35-40\% of points are distributed over average daily attendance, the My Voice, My School 5Essentials survey results, English langauge learner performance on the ACCESS assessment, and school performance on submitting and maintaining data in IMPACT. Performance Points on MAP are allocated over four broad metrics: National School Growth Percentile, priority group performance measured by the National School Growth Percentile, Percentage of students meeting or exceeding typical growth, and National School Attainment Percentile. 
\end{abstract}







\section{The SQRP in a Nutshell}
\newthought{The SQRP is a weighted points system} that assigns from one to five points to each of (up to) 19 measures. Fifteen of the measures are derived from MAP performance in reading and mathematics (note though \emph{not} general science) with growth-based measures using spring-to-spring calculations. The MAP-based measures recieve between 60 and 65\% of weight in calculating the weighted points earned.  The remainder of the weights (35-40\%) are allocated to ELL growth, attendence, a student-family satisfaction survey, and data quality in IMPACT.  

\subsection{Assigning Points}
On any given performance indicator one to five points are allocated based on the school's performance.  The allocations depend on cut-offs or performance bands, which depend, perhaps obviously, on the indicator being considered. The points received are then multiplied by a weight between 0 and 20\%. Some of the indicators have fixed weights (e.g., the DQI, ADA, and \emph{5Essentials}); however, the academic measures have \emph{\textbf{variable}} weights.\footnote{Essentially the weights vary by the composition of school's priority groups.} 

\subsection{Weighting}

Table \ref{table:weights} provides the weights in more detail.  As noted already NWEA MAP-based indicators comprise two-thirds of the weighted score.  MAP-based indicators can be broken out into four broad types: nationally normed school-level growth across all students, nationally normed school level growth across priority group students,\footnote{\textbf{Priority group students} are those that are African American, Latina/o, have an IEP, or are designated English language learners (ELL). A school's priority group students are evaluated only if there are at least 30 such students.} the percent of students meeting or exceeding typical growth for their peer groups (i.e., the so-called NWEA Growth Targets), and finally the school's overall RIT score attainement relative to national norms. 

\begin{margintable}\scriptsize  
  \begin{tabular}{|l|r|l|}
    \hline
    \multicolumn{2}{|c|}{\textbf{Indicator Group}} &  \textbf{Weight} \\
    \hline
     & & \\
    \multirow{4}{*}{MAP-based} & {Nat'l School Growth \%ile} & 25-40\% \\
     & Priority Group Nat'l Growth \%ile                    & 0-10\%  \\
     & \% Meets/Exceeds Growth Norm                      & 10\%    \\
     & Nat'l School Attainment \%ile                          & 15\%    \\
     & & \\
    \hline
    \multicolumn{2}{|r|}{ACCESS \% Making Sufficient Progress} & 0-5\%   \\ 
    \multicolumn{2}{|r|}{Average Daily Attendance}              & 20\%    \\
    \multicolumn{2}{|r|}{My Voice, My School 5Essentials}       & 10\%    \\
    \multicolumn{2}{|r|}{Data Quality Index}                    & 5\%    \\
    \hline
  \end{tabular}
  \caption{Elementary Schools SQRP Component Weights}
  \label{table:weights}
\end{margintable}

English language learner performance on the ACCESS assessment comprises 0-5\% percent of the weight.  Average daily attendence is worth 20\%, the \emph{5Essentials} survey is worth 10\%, and the CPS \emph{Data Quality Index} is worth the remaing 5\% of weight.




\subsection{Policy Scoring}

Calculating the SQRP is a multistep process.  Points are assigned to each of 19 potential indicators based on a school's results on each of the indicators. Note however that not all indicators apply to all schools (e.g., $2nd$ grade specific indicators, of which their are two,  would not apply a to 5-8 middle school). Consequently, the indicator itself, as well as the proper weight, must be calculated.\footnote{How one calculates each indicator is outside the scope of this brief memo, suffice it say that some of the calculations are convoluted.  KIPP Chicago Schools has created a tool for performing each calculation by relying on publically avialable information from NWEA, but it provides only a first approximation for the National School Growth Percentile indicator.} Once each indicator has been calculated up to 5 points are assigned to the metric. For MAP- and ACCESS-based metrics, schools can have points deducted for low student participation rates on the assessments. The (perhaps participation adjusted) points are multiplied by a weight for each indicator. Figure \ref{fig:simple} provides a simplified schematic showing how a single indicator's weighted points are calculated.

% Define block styles
\tikzstyle{decision} = [diamond, draw, 
    text width=4.5em, text badly centered, node distance=2.5cm, inner sep=0pt]
\tikzstyle{block} = [rectangle, draw, fill=blue!20, 
    text width=5em, text centered, rounded corners, minimum height=4em]
\tikzstyle{line} = [draw, -latex']




\begin{figure*}\footnotesize
\begin{tikzpicture}[scale=1, node distance = 3.5cm, auto]
    % Place nodes
    \node (ind_title) {\textbf{Indicator}};
    \node [,right of=ind_title] (results_title) {\textbf{School's result}};
    \node [,right of=results_title] (points_title) {\textbf{Points}};
    \node [,right of=points_title] (weight_title) {\textbf{Weight}};
    \node [,right of=weight_title] (wpoints_title) {\textbf{Weighted Points}};
    
    \node [block, below of=ind_title, node distance=1cm]  (indicator) {School Growth \%ile Reading};
    \node [block, right of=indicator, fill=orange!20] (result) {95th};
    \node [block, right of=result] (points) {5};
    \node [block, right of=points] (weight) {17.5\%};
    \node [block, right of=weight] (wpoints) {0.875};
    % Draw edges
    \path [line] (indicator) -- node {{\tiny calculate}} (result);
    \path [line] (result) -- node {{\tiny table look-up}} (points);
    \path [line] (points) -- node {{\footnotesize $\times$}} (weight);
    \path [line] (weight) -- node {{\footnotesize $=$}} (wpoints);
\end{tikzpicture}
\caption{Simplified schematic of weighted policy scoring}
\label{fig:simple}
\end{figure*}

The process of calculating weighted points is performed for each indicator and then all the weighted points are summed up resulting in a final score that also varies from 0 to 5. Figure \ref{fig:detail} lays out the process for a middle school serving grades 5-8 and having only two priority groups:  African American students and students with an IEP.  Once the SQRP weighted score is known the school can be assinged to one of the five tiers in the \emph{School Quality Rating}.  Table \ref{table:tiers} provides the mapping from a school's weighted score to School Quality Rating tier.
\begin{margintable}\scriptsize  
  \begin{tabular}{|c|c|}
    \hline
    \textbf{Weighted Score} &  \textbf{School Quality Rating Tier} \\
    \hline
    4.0 $+$     & Tier 1 \\
    3.5 - 3.9   & Tier 2 \\
    3.0 - 3.4   & Tier 3 \\
    2.0 - 2.9   & Tier 4 \\
    $<$ 2.0     & Tier 5 \\
    \hline
  \end{tabular}
  \caption{Mapping of a schools overall weighted score to School Quality Rating Tier}
  \label{table:tiers}
\end{margintable}


\begin{figure}\tiny
\begin{tikzpicture}[scale=1, node distance = 2.4cm, auto]
    % Place nodes
    \node (ind_title) {\textbf{Indicator}};
    \node [,right of=ind_title] (results_title) {\textbf{School's result}};
    \node [,right of=results_title] (points_title) {\textbf{Points}};
    \node [,right of=points_title] (weight_title) {\textbf{Weight}};
    \node [,right of=weight_title] (wpoints_title) {\textbf{Weighted Points}};
    
    \node [block, below of=ind_title, node distance=.75cm]  (indicator) {School Growth \%ile Reading};
    \node [block, right of=indicator, fill=orange!20] (result) {96th};
    \node [block, right of=result] (points) {5};
    \node [block, right of=points] (weight) {17.5\%};
    \node [block, right of=weight] (wpoints) {0.875};
    % Draw edges
    \path [line] (indicator) -- (result);
    \path [line] (result) --  (points);
    \path [line] (points) -- node {{\footnotesize $\times$}} (weight);
    \path [line] (weight) -- node {{\footnotesize $=$}} (wpoints);
    
    %Math
    \node [block, below of=indicator, node distance=1cm]  (indicator2) {School Growth \%ile Math};
    \node [block, right of=indicator2, fill=orange!20] (result2) {91st};
    \node [block, right of=result2] (points2) {5};
    \node [block, right of=points2] (weight2) {17.5\%};
    \node [block, right of=weight2] (wpoints2) {0.875};
    % Draw edges
    \path [line] (indicator2) -- (result2);
    \path [line] (result2) --  (points2);
    \path [line] (points2) --  (weight2);
    \path [line] (weight2) --  (wpoints2);
    
    %AA Reading
    \node [block, below of=indicator2, node distance=1cm]  (indicator3) {Priority AA School Growth \%ile Reading};
    \node [block, right of=indicator3, fill=orange!20] (result3) {97th};
    \node [block, right of=result3] (points3) {5};
    \node [block, right of=points3] (weight3) {1.25\%};
    \node [block, right of=weight3] (wpoints3) {0.063};
    % Draw edges
    \path [line] (indicator3) -- (result3);
    \path [line] (result3) --  (points3);
    \path [line] (points3) --  (weight3);
    \path [line] (weight3) --  (wpoints3);
    
     %AA Math
    \node [block, below of=indicator3, node distance=1.2cm]  (indicator4) {Priority AA School Growth \%ile Math};
    \node [block, right of=indicator4, fill=orange!20] (result4) {88th};
    \node [block, right of=result4] (points4) {5};
    \node [block, right of=points4] (weight4) {1.25\%};
    \node [block, right of=weight4] (wpoints4) {0.063};
    % Draw edges
    \path [line] (indicator4) -- (result4);
    \path [line] (result4) --  (points4);
    \path [line] (points4) --  (weight4);
    \path [line] (weight4) --  (wpoints4);
    
        %IEP Reading
    \node [block, below of=indicator4, node distance=1.2cm]  (indicator5) {Priority IEP School Growth \%ile Reading};
    \node [block, right of=indicator5, fill=orange!20] (result5) {57th};
    \node [block, right of=result5] (points5) {4};
    \node [block, right of=points5] (weight5) {1.25\%};
    \node [block, right of=weight5] (wpoints5) {0.050};
    % Draw edges
    \path [line] (indicator5) -- (result5);
    \path [line] (result5) --  (points5);
    \path [line] (points5) --  (weight5);
    \path [line] (weight5) --  (wpoints5);
    
     %IEP Math
    \node [block, below of=indicator5, node distance=1.2cm]  (indicator6) {Priority IEP School Growth \%ile Math};
    \node [block, right of=indicator6, fill=orange!20] (result6) {69th};
    \node [block, right of=result6] (points6) {4};
    \node [block, right of=points6] (weight6) {1.25\%};
    \node [block, right of=weight6] (wpoints6) {0.050};
    % Draw edges
    \path [line] (indicator6) -- (result6);
    \path [line] (result6) --  (points6);
    \path [line] (points6) --  (weight6);
    \path [line] (weight6) --  (wpoints6);

     %% M/E
    \node [block, below of=indicator6, node distance=1cm]  (indicator7) {\% M/E Typical Growth};
    \node [block, right of=indicator7, fill=orange!20] (result7) {79\%};
    \node [block, right of=result7] (points7) {5};
    \node [block, right of=points7] (weight7) {10\%};
    \node [block, right of=weight7] (wpoints7) {0.500};
    % Draw edges
    \path [line] (indicator7) -- (result7);
    \path [line] (result7) --  (points7);
    \path [line] (points7) --  (weight7);
    \path [line] (weight7) --  (wpoints7);
    
    %% Reading Attainment
    \node [block, below of=indicator7, node distance=1cm]  (indicator8) {National School Attainment \%ile Reading};
    \node [block, right of=indicator8, fill=orange!20] (result8) {71st};
    \node [block, right of=result8] (points8) {4};
    \node [block, right of=points8] (weight8) {7.5\%};
    \node [block, right of=weight8] (wpoints8) {0.300};
    % Draw edges
    \path [line] (indicator8) -- (result8);
    \path [line] (result8) --  (points8);
    \path [line] (points8) --  (weight8);
    \path [line] (weight8) --  (wpoints8);
    
    %% MAth Attainment
    \node [block, below of=indicator8, node distance=1.2cm]  (indicator9) {National School Attainment \%ile MAth};
    \node [block, right of=indicator9, fill=orange!20] (result9) {50th};
    \node [block, right of=result9] (points9) {3};
    \node [block, right of=points9] (weight9) {7.5\%};
    \node [block, right of=weight9] (wpoints9) {0.225};
    % Draw edges
    \path [line] (indicator9) -- (result9);
    \path [line] (result9) --  (points9);
    \path [line] (points9) --  (weight9);
    \path [line] (weight9) --  (wpoints9);
    
    %% ADA
    \node [block, below of=indicator9, node distance=1cm]  (indicator10) {ADA};
    \node [block, right of=indicator10, fill=orange!20] (result10) {95};
    \node [block, right of=result10] (points10) {4};
    \node [block, right of=points10] (weight10) {20\%};
    \node [block, right of=weight10] (wpoints10) {0.800};
    % Draw edges
    \path [line] (indicator10) -- (result10);
    \path [line] (result10) --  (points10);
    \path [line] (points10) --  (weight10);
    \path [line] (weight10) --  (wpoints10);
    
    %% 5Essentials
    \node [block, below of=indicator10, node distance=1cm]  (indicator11) {5Essentials};
    \node [block, right of=indicator11, fill=orange!20] (result11) {O};
    \node [block, right of=result11] (points11) {4};
    \node [block, right of=points11] (weight11) {10\%};
    \node [block, right of=weight11] (wpoints11) {0.400};
    % Draw edges
    \path [line] (indicator11) -- (result11);
    \path [line] (result11) --  (points11);
    \path [line] (points11) --  (weight11);
    \path [line] (weight11) --  (wpoints11);
    
     %% DQI
    \node [block, below of=indicator11, node distance=1cm]  (indicator12) {DQI};
    \node [block, right of=indicator12, fill=orange!20] (result12) {92\%};
    \node [block, right of=result12] (points12) {3};
    \node [block, right of=points12] (weight12) {5\%};
    \node [block, right of=weight12] (wpoints12) {0.150};
    % Draw edges
    \path [line] (indicator12) -- (result12);
    \path [line] (result12) --  (points12);
    \path [line] (points12) --  (weight12);
    \path [line] (weight12) --  (wpoints12);
    
    % sum path
    \path [line] (wpoints) -- node {$+$} (wpoints2);
    \path [line] (wpoints2) -- node {$+$} (wpoints3);
    \path [line] (wpoints3) -- node {$+$} (wpoints4);
    \path [line] (wpoints4) -- node {$+$} (wpoints5);
    \path [line] (wpoints5) -- node {$+$} (wpoints6);
    \path [line] (wpoints6) -- node {$+$} (wpoints7);
    \path [line] (wpoints7) -- node {$+$} (wpoints8);
    \path [line] (wpoints8) -- node {$+$} (wpoints9);
    \path [line] (wpoints9) -- node {$+$} (wpoints10);
    \path [line] (wpoints10) -- node {$+$} (wpoints11);
    \path [line] (wpoints11) -- node {$+$} (wpoints12);
    
    
    \node [block, below of=wpoints12, fill=yellow!20, node distance=1.2cm] (policy) {\textbf{4.4 Tier 1}};
    \path [line] (wpoints12) -- node {$=$} (policy);
    
    
\end{tikzpicture}
\caption{More detailed schematic of weighted policy scoring process for an elementary school with grades 5-8 and with more than 30 African American students as well as more than 30 students with IEPs (indicating two priority groups having sufficent numbers to enter int SQRP calculation. Note that all NWEA MAP growth calculations (i.e., school-level growth as well as individual growth is calculated spring-to-spring.))}\label{fig:detail}
\end{figure}

\section{NWEA MAP Components in Detail}
As you have likely noticed above, the MAP-based components are broken out into many---in fact 15---seperate indicators which can be grouped into three categories by the type of indicator to be calculated (weights in parenthesis):

\begin{enumerate}
  \item National School Growth Percentile on NWEA MAP
  \begin{itemize}
    \item Reading Grades 3-8 (12.5\%)*
    \item Math Assessment Grades 3-8 (12.5\%)*
    \item African American Priority Group Reading (1.25\%)
    \item Hispanic Priorty Group Reading (1.25\%)
    \item ELL Priority Group Reading(1.25\%)
    \item Diverse Learners (IEP) Priority Group  Reading (1.25\%)
    \item African American Priority Group Math (1.25\%)
    \item Hispanic Priority Group Math (1.25\%)
    \item ELL Priority Group Math (1.25\%)
    \item Diverse Learners (IEP) Priority Group Math (1.25\%)
  \end{itemize}
  \item Percentage of Students Meeting or Exceeding National Average Growth Norms, Grades 3-8 (10\%)*
  \item National School Attainment Percentile
  \begin{itemize} 
    \item NWEA Reading Grades 3-8 (5\%) *
    \item  NWEA Math Grades 3-8 (5\%)*
    \item Reading Grade 2 (2.5\%)
    \item  Math Grade 2 (2.5\%)
  \end{itemize}
\end{enumerate}

All indicators require at least 10 students to have taken the MAP assessment, with the exception of the the Priority Groups, which require at least 30 students. The asterisks (*) denote indicators that if missing result in no SQRP rating being calculated for a school.  If any of the other indicators do not have the requisite number of students, then their weights are allocated to the the most similar, ``higher level'' measure.  For example, if a priorty group has less than 30 students then the weight for that group's School Growth Percentile ios added to the whole school's School Growth Percentile. How then is each indicator type calculated?

\begin{description}
\item[National School Growth Percentile] is to schools what growth magnitude is to students.  Most simply put, for each  subject, the amount of growth achived by the school is calcuated using average RIT scores over two MAP assessments and this growth is compared to the expected growth derived from a nationally normed sample of schools. The school is assigned a percentile representing where it would fall on the national distribution of all schools with the same pre-test (prior spring) average RIT score.\footnote{The average RIT score is specifically a weighted average of each grade's average RIT score where the weights are the proportion of students in each grade (i.e, weight $=\frac{\# \mbox{ students in grade}}{\# \mbox{ students in school}}$).}

For the purposes ofthe SQRP, this measure is calculated across all students in grades 3-8 as well across only students that are  members of a priority group. 
\item[Percent of Students Meeting or Exceeding Typical Growth] should be very familiar to the reader.  It is simply the proportion of students meeting or exceeding their individual expected RIT growth over the number of students tested.  This metric is calculated using spring-to-spring expected growth and for students in grades 2 - 8.
\item[National School Attainment Percentile] is to schools what national percentile rank is to students. It compares the average spring RIT score of the school versus the average spring score of all schools nationally and assigns a percentile rank representing where the school would fall on the national distribution of all schools.  Unlike the National School Growth Percentile, this indicator is not adjusted for prior spring performance.  The percentile is calculated separately for all students in grades 3-8 and all students in grade 2.  Note that this indicator combines math and reading \emph{tests} and thus is an indicator of the percentage of tests take rather than the percentage of students.  
\end{description}



\end{document}
